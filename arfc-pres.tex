%        File: arfc-beamer.tex
%     Created: Sun May 5 10:00 PM 2013 C
%


%\documentclass[11pt,handout]{beamer}
\documentclass[9pt]{beamer}
\usetheme[white]{Illinois}
%\title[short title]{long title}
\title[Application of hafnium hydride control rod to large sodium cooled fast breeder reactor]{Review of Application of hafnium hydride control rod to large sodium cooled fast breeder reactor}
%\subtitle[short subtitle]{long subtitle}
\subtitle[]{}
%\author[short name]{long name}
\author[Kazumi Ikeda, Hiroyuki Moriwaki, Yoshiyuki Ohkubo, Tomohiko Iwasaki, Kennji Konashi]{Kazumi Ikeda, Hiroyuki Moriwaki, Yoshiyuki Ohkubo, Tomohiko Iwasaki, Kennji Konashi}
%\date[short date]{long date}
\date[01.23.2018]{January 23, 2018}
%\institution[short name]{long name}
\institute[]{}

%%%% Acronym support

\usepackage[acronym,toc]{glossaries}
\include{acros}

\makeglossaries

%\usepackage{bbding}
\usepackage{amsfonts}
\usepackage{adjustbox}
\usepackage{amsmath}
\usepackage{xspace}
\usepackage{graphicx}
\usepackage{subfigure}
\usepackage{booktabs} % nice rules for tables
\usepackage{microtype} % if using PDF
\usepackage{bigints}
\DeclareMathOperator{\erf}{erf}
%I need some complimentary error funcitons... 
\DeclareMathOperator{\erfc}{erfc}
%page numbers
\setbeamertemplate{footline}[page number]
\setbeamertemplate{caption}[numbered]
%Those icons in the references are terrible looking
\setbeamertemplate{bibliography item}[text]


%try to get rid of header on title page\dots
\makeatletter
    \newenvironment{withoutheadline}{
        \setbeamertemplate{headline}[default]
        \def\beamer@entrycode{\vspace*{-\headheight}}
    }{}
\makeatother


\usepackage{booktabs} % nice rules (thick lines) for tables
\usepackage{microtype} % improves typography for PDF
\usepackage{xspace}
\usepackage{tabularx}
\usepackage[affil-it]{authblk}
\usepackage{tikz}

\usepackage{tikz}
\usetikzlibrary{positioning, arrows, decorations, shapes}

\usetikzlibrary{shapes.geometric,arrows}
\tikzstyle{process} = [rectangle, rounded corners, minimum width=3cm, minimum height=1cm,text centered, draw=black, fill=blue!30]
\tikzstyle{object} = [ellipse, rounded corners, minimum width=3cm, minimum height=1cm,text centered, draw=black, fill=green!30]
\tikzstyle{arrow} = [thick,->,>=stealth]

\usepackage{cleveref}
\usepackage{datatool}
\newcolumntype{b}{X}
\newcolumntype{s}{>{\hsize=.5\hsize}X}
\newcolumntype{m}{>{\hsize=.75\hsize}X}

\newcommand{\Cyclus}{\textsc{Cyclus}\xspace}%
\graphicspath{ {images/} }
\usetikzlibrary{positioning, arrows, decorations, shapes }

\begin{document}
%%%%%%%%%%%%%%%%%%%%%%%%%%%%%%%%%%%%%%%%%%%%%%%%%%%%%%%%%%%%%
%% From uw-beamer Here's a handy bit of code to place at 
%% the beginning of your presentation (after \begin{document}):
\newcommand*{\alphabet}{ABCDEFGHIJKLMNOPQRSTUVWXYZabcdefghijklmnopqrstuvwxyz}
\newlength{\highlightheight}
\newlength{\highlightdepth}
\newlength{\highlightmargin}
\setlength{\highlightmargin}{2pt}
\settoheight{\highlightheight}{\alphabet}
\settodepth{\highlightdepth}{\alphabet}
\addtolength{\highlightheight}{\highlightmargin}
\addtolength{\highlightdepth}{\highlightmargin}
\addtolength{\highlightheight}{\highlightdepth}
\newcommand*{\Highlight}{\rlap{\textcolor{HighlightBackground}{\rule[-\highlightdepth]{\linewidth}{\highlightheight}}}}
%%%%%%%%%%%%%%%%%%%%%%%%%%%%%%%%%%%%%%%%%%%%%%%%%%%%%%%%%%%%%
%%--------------------------------%%
\begin{withoutheadline}
\frame{
  \titlepage
}
\end{withoutheadline}

%%--------------------------------%%
\AtBeginSection[]{
\begin{frame}
  \frametitle{Outline}
  \tableofcontents[currentsection]
\end{frame}
}

\section{Review of Paper}
\begin{frame}
\frametitle{Motivation}

\begin{itemize}
  \item Current \gls{SFR} designs have enriched B-10 boron carbide
  \begin{itemize}
    \item Swell due to accumulation of He and Li produced by (n,$\alpha$)
    \item He buildup in the gas plenum
    \item Degradation of control rod worth (lifetime \textasciitilde 2 years)
  \end{itemize}
  \item Need for a control rod with a longer lifetime 
\end{itemize}

\end{frame}

\begin{frame}
\frametitle{Hafnium as an absorber - 1/3}
Comparable absorption cross section
\begin{figure}[htbp!]
  \begin{center}
      \includegraphics[scale=0.25]{./images/axs.png}
  \end{center}
  \caption{Absorption cross section comparison of various forms}
  \label{fig:axs}
\end{figure}

\end{frame}

\begin{frame}
\frametitle{Hafnium as an absorber - 2/3}
Continued absorption integrity and low induced radioactivity after irradiation.

\begin{figure}[htbp!]
  \begin{center}
      \includegraphics[scale=0.2]{./images/decay_chain.png}
  \end{center}
  \caption{Nuclear transmutation of halfnium nuclides by capture. Ta-181 is stable and Hf-182
            has a half life of 8.9e6 years}
  \label{fig:dec}
\end{figure}
\end{frame}

\begin{frame}
\frametitle{Hafnium as an absorber - 3/3}
Other peripheral beneficial characteristics of halfnium include:
\begin{itemize}
  \item Malleability, ductility
  \item Rich reserve of raw materials
  \item Resistance to chemical activity
  \item Corrosion resistance
  \item High melting temperature
  \item Stable at high temperature and pressure
\end{itemize}
\end{frame}

\section{Methodology}

\begin{frame}
\frametitle{Codes and Libraries}
\begin{itemize}
\item Diffusion Code (TRISTAN)
  \begin{itemize}
    \item verified with CITATION from \gls{ORNL}.
  \end{itemize}
\item Perturbation Code (TRI-PERT)
  \begin{itemize}
  \item nuclear characteristics
  \end{itemize}
\item Effects of reactor constants and heterogeneity of control rod structure (GMVP)
\item Depletion of hafnium (JFS3-J3.3)
\item Heating density of hafnium hydride absorber (MCNP-5)
\item Nuclear Data (ENDL92 for Hf and ENDF-VI for others)
\end{itemize}
\end{frame}

\begin{frame}
\frametitle{Core simulation with hafnium control rods}
Hafnium hydride control rods replaced the boron carbide in the
\gls{JSFR} for comparison, with U-TRU mixed oxide fuel.

\begin{figure}[htbp!]
  \begin{center}
      \includegraphics[scale=0.18]{./images/core.png}
  \end{center}
  \caption{Core diagram of the \gls{JSFR}.}
  \label{fig:core}
\end{figure}

\end{frame}


%%--------------------------------%%
%%--------------------------------%%
\begin{frame}[allowframebreaks]
  \frametitle{References}
  \bibliographystyle{abbrv}
  {\footnotesize \bibliography{bibliography} }
\end{frame}

%%--------------------------------%%


\end{document}



