%        File: arfc-beamer.tex
%     Created: Sun May 5 10:00 PM 2013 C
%


%\documentclass[11pt,handout]{beamer}
\documentclass[9pt]{beamer}
\usetheme[white]{Illinois}
%\title[short title]{long title}
\title[Application of hafnium hydride control rod to large sodium cooled fast breeder reactor]{Review of Application of hafnium hydride control rod to large sodium cooled fast breeder reactor}
%\subtitle[short subtitle]{long subtitle}
\subtitle[]{}
%\author[short name]{long name}
\author[Kazumi Ikeda, Hiroyuki Moriwaki, Yoshiyuki Ohkubo, Tomohiko Iwasaki, Kennji Konashi]{Kazumi Ikeda, Hiroyuki Moriwaki, Yoshiyuki Ohkubo, Tomohiko Iwasaki, Kennji Konashi}
%\date[short date]{long date}
\date[01.23.2018]{January 23, 2018}
%\institution[short name]{long name}
\institute[]{}

%%%% Acronym support

\usepackage[acronym,toc]{glossaries}
\include{acros}

\makeglossaries

%\usepackage{bbding}
\usepackage{amsfonts}
\usepackage{adjustbox}
\usepackage{amsmath}
\usepackage{xspace}
\usepackage{graphicx}
\usepackage{subfigure}
\usepackage{booktabs} % nice rules for tables
\usepackage{microtype} % if using PDF
\usepackage{bigints}
\DeclareMathOperator{\erf}{erf}
%I need some complimentary error funcitons... 
\DeclareMathOperator{\erfc}{erfc}
%page numbers
\setbeamertemplate{footline}[page number]
\setbeamertemplate{caption}[numbered]
%Those icons in the references are terrible looking
\setbeamertemplate{bibliography item}[text]


%try to get rid of header on title page\dots
\makeatletter
    \newenvironment{withoutheadline}{
        \setbeamertemplate{headline}[default]
        \def\beamer@entrycode{\vspace*{-\headheight}}
    }{}
\makeatother


\usepackage{booktabs} % nice rules (thick lines) for tables
\usepackage{microtype} % improves typography for PDF
\usepackage{xspace}
\usepackage{tabularx}
\usepackage[affil-it]{authblk}
\usepackage{tikz}

\usepackage{tikz}
\usetikzlibrary{positioning, arrows, decorations, shapes}

\usetikzlibrary{shapes.geometric,arrows}
\tikzstyle{process} = [rectangle, rounded corners, minimum width=3cm, minimum height=1cm,text centered, draw=black, fill=blue!30]
\tikzstyle{object} = [ellipse, rounded corners, minimum width=3cm, minimum height=1cm,text centered, draw=black, fill=green!30]
\tikzstyle{arrow} = [thick,->,>=stealth]

\usepackage{cleveref}
\usepackage{datatool}
\newcolumntype{b}{X}
\newcolumntype{s}{>{\hsize=.5\hsize}X}
\newcolumntype{m}{>{\hsize=.75\hsize}X}

\newcommand{\Cyclus}{\textsc{Cyclus}\xspace}%
\newcommand{\hfh}{$HfH_{x}$\xspace}
\newcommand{\bc}{$B_4C$\xspace}
\graphicspath{ {images/} }
\usetikzlibrary{positioning, arrows, decorations, shapes }

\begin{document}
%%%%%%%%%%%%%%%%%%%%%%%%%%%%%%%%%%%%%%%%%%%%%%%%%%%%%%%%%%%%%
%% From uw-beamer Here's a handy bit of code to place at 
%% the beginning of your presentation (after \begin{document}):
\newcommand*{\alphabet}{ABCDEFGHIJKLMNOPQRSTUVWXYZabcdefghijklmnopqrstuvwxyz}
\newlength{\highlightheight}
\newlength{\highlightdepth}
\newlength{\highlightmargin}
\setlength{\highlightmargin}{2pt}
\settoheight{\highlightheight}{\alphabet}
\settodepth{\highlightdepth}{\alphabet}
\addtolength{\highlightheight}{\highlightmargin}
\addtolength{\highlightdepth}{\highlightmargin}
\addtolength{\highlightheight}{\highlightdepth}
\newcommand*{\Highlight}{\rlap{\textcolor{HighlightBackground}{\rule[-\highlightdepth]{\linewidth}{\highlightheight}}}}
%%%%%%%%%%%%%%%%%%%%%%%%%%%%%%%%%%%%%%%%%%%%%%%%%%%%%%%%%%%%%
%%--------------------------------%%
\begin{withoutheadline}
\frame{
  \titlepage
}
\end{withoutheadline}

%%--------------------------------%%
\AtBeginSection[]{
\begin{frame}
  \frametitle{Outline}
  \tableofcontents[currentsection]
\end{frame}
}

\section{Review of Paper}
\begin{frame}
\frametitle{Motivation}

\begin{itemize}
  \item Current \gls{SFR} designs have enriched B-10 boron carbide
  \begin{itemize}
    \item Swell due to accumulation of He and Li produced by (n,$\alpha$)
    \item He buildup in the gas plenum
    \item Degradation of control rod worth (lifetime \textasciitilde 2 years)
  \end{itemize}
  \item Need for a control rod with a longer lifetime 
\end{itemize}

\end{frame}

\begin{frame}
\frametitle{Hafnium as an absorber - 1/3}
Comparable absorption cross section
\begin{figure}[htbp!]
  \begin{center}
      \includegraphics[scale=0.25]{./images/axs.png}
  \end{center}
  \caption{Absorption cross section comparison of various forms}
  \label{fig:axs}
\end{figure}

\end{frame}

\begin{frame}
\frametitle{Hafnium as an absorber - 2/3}
Continued absorption integrity and low induced radioactivity after irradiation.

\begin{figure}[htbp!]
  \begin{center}
      \includegraphics[scale=0.2]{./images/decay_chain.png}
  \end{center}
  \caption{Nuclear transmutation of hafnium nuclides by capture. Ta-181 is stable and Hf-182
            has a half life of 8.9e6 years}
  \label{fig:dec}
\end{figure}
\end{frame}

\begin{frame}
\frametitle{Hafnium as an absorber - 3/3}
Other peripheral beneficial characteristics of hafnium include:
\begin{itemize}
  \item Malleability, ductility
  \item Rich reserve of raw materials
  \item Resistance to chemical activity
  \item Corrosion resistance
  \item High melting temperature
  \item Stable at high temperature and pressure
\end{itemize}
\end{frame}

\section{Methodology}

\begin{frame}
\frametitle{Codes and Libraries}
\begin{itemize}
\item Unified 70-group fast set of group constants ADJ2000R \cite{hazama_development_2002}
\item Diffusion Code (TRISTAN)
  \begin{itemize}
    \item verified with CITATION from \gls{ORNL}.
  \end{itemize}
\item Perturbation Code (TRI-PERT) 
  \begin{itemize}
  \item nuclear characteristics
  \end{itemize}
\item Effects of reactor constants and heterogeneity of control rod structure (GMVP) \cite{nagaya_mvp/gmvp_2005}
\item Depletion of hafnium (JFS3-J3.3)
\item Heating density of hafnium hydride absorber (MCNP-5) \cite{mcnp_monte_2003}
\item Nuclear Data (ENDL92 for Hf and ENDF-VI for others)
\end{itemize}
\end{frame}

\begin{frame}
\frametitle{Core simulation with hafnium control rods}
Hafnium hydride control rods replaced the boron carbide in the
\gls{JSFR} \cite{ogura_conceptual_2009} for comparison, with U-TRU mixed oxide fuel \cite{ikeda_application_2014}.

\begin{figure}[htbp!]
  \begin{center}
      \includegraphics[scale=0.15]{./images/core.png}
  \end{center}
  \caption{Core diagram of the \gls{JSFR}.}
  \label{fig:core}
\end{figure}

\end{frame}


\section{Results}


\begin{frame}
\frametitle{Reactivity of \hfh control rods}
Reactivity similar to 80\% B-10 enriched boron carbide rods.
Enhanced control rod reactivity with hydrogen moderation of neutrons.
\begin{figure}[htbp!]
  \begin{center}
      \includegraphics[scale=0.5]{./images/spectrum.png}
  \end{center}
  \caption{Neutron spectrum of the hafnium hydride control rod}
  \label{fig:spec}
\end{figure}
\end{frame}

\begin{frame}
\frametitle{\hfh control rod irradiation and performance change - 1/2}
Neutron absorption performance degrades very little with irradiation
because the produced isotopes are also absorbers.
\begin{figure}[htbp!]
  \begin{center}
      \includegraphics[scale=0.5]{./images/irrad_isotope.png}
  \end{center}
  \caption{Change of hafnium isotope ratio in the control rod during irradiation.}
  \label{fig:irrad_iso}
\end{figure}
\end{frame}

\begin{frame}
\frametitle{\hfh control rod irradiation and performance change - 2/2}
Reactivity degradation is 4\% after 2400 \gls{EFPD}
\begin{figure}[htbp!]
  \begin{center}
      \includegraphics[scale=0.5]{./images/irrad_reac.png}
  \end{center}
  \caption{Capture reaction rate and reactivity of hafnium hydride control rod.}
  \label{fig:irrad_reac}
\end{figure}
\end{frame}

\begin{frame}
\frametitle{Heat profile in \hfh  control rod}
Neutron-induced energy of the absorber is approximately twice as large as \bc
Limit of 800$^\circ$C to prevent hydrogen desorption.
Maximum absorber temperature = 631$^\circ$C
\begin{figure}[htbp!]
  \begin{center}
      \includegraphics[scale=0.28]{./images/th.png}
  \end{center}
  \caption{Thermal hydraulic characteristics in the hafnium hydride control rod deployed reactor.}
  \label{fig:th}
\end{figure}
\end{frame}

\begin{frame}
\frametitle{Changes in reactor with \hfh control rod}
Nuclear characteristics as good as or better than with \bc.
\begin{figure}[htbp!]
  \begin{center}
      \includegraphics[scale=0.25]{./images/reac.png}
  \end{center}
  \caption{Nuclear characteristics of the reactors with the hafnium hydride and the boron carbide absorber.}
  \label{fig:reac}
\end{figure}
\end{frame}

\begin{frame}
\frametitle{Heat density of \hfh}
Twice the heat generation of \bc, but peak of heat densities are comparable
due to hydrogen moderation and gamma-ray transport.
\begin{figure}[htbp!]
  \begin{center}
      \includegraphics[scale=0.3]{./images/gamma.png}
  \end{center}
  \caption{Effect of gamma transport in the linear heat rate of \hfh absorber.}
  \label{fig:reac}
\end{figure}
\end{frame}


\begin{frame}
\frametitle{Design requirements of \gls{SFR}}
    \begin{table}[h]
      \centering 
      \begin{tabularx}{\textwidth}{mb}
      \hline
      Requirement & Benefit from using \hfh \\ 
      \hline
      Intrinsic negative feedback &     \\ 
      Na void reactivity $\leq$ \$6 & \$5.1\\ 
      Coolant flow &  8\% less at cladding temperature limit  \\
      Shutdown abilities &  \\ 
      Hydrogen desorption & future work\\ 
      Off-normal events & future work \\ 
      \hline
      \end{tabularx}
      \caption {Benefits of using \hfh in \gls{SFR} relative to \bc}
      \label{tab:ben}
    \end{table}
\end{frame}


\begin{frame}
\frametitle{Important Results and Conclusion}
\begin{itemize}
  \item Longer lifetime due to hafnium isotopes after neutron absorption
  \item Hydrogen increase of reactivity
  \item 9 years lifetime (compared with 1-2 year for \bc)
  \item Better reactor parameters (flow rate, safety margin, breeding ratio etc.)
  \item Better thermal heat profile during operation
\end{itemize}
\end{frame}

\section{Critique}

\begin{frame}
\frametitle{Missing}
\begin{itemize}
  \item Quantification of intrinsic negative feedback
  \item 
\end{itemize}
\end{frame}

\begin{frame}
\frametitle{Relevance}
\begin{itemize}
  \item
\end{itemize}
\end{frame}



\begin{frame}
\frametitle{Novelty}
\begin{itemize}
  \item
\end{itemize}
\end{frame}


\begin{frame}
\frametitle{Technical Detail}
\begin{itemize}
  \item
\end{itemize}
\end{frame}


\begin{frame}
\frametitle{Analytic Rigour}
\begin{itemize}
  \item
\end{itemize}
\end{frame}


\begin{frame}
\frametitle{Verifiability (Reproducibility)}
\begin{itemize}
  \item
\end{itemize}
\end{frame}


\begin{frame}
\frametitle{Clarity}
\begin{itemize}
  \item
\end{itemize}
\end{frame}


\begin{frame}
\frametitle{Miscellaneous}
\begin{itemize}
  \item
\end{itemize}
\end{frame}

\section{Extension}


\begin{frame}
\frametitle{Future Work}
\begin{itemize}
  \item hydrogen desorption in operating conditions
  \item accident scenarios
    leads to higher core / control rod temperature -> desorption -> less reactivity -> worse
  \item uncertainty quantification
  \item Swelling of control rod
  \item 
\end{itemize}

\end{frame}



%%--------------------------------%%
%%--------------------------------%%
\begin{frame}[allowframebreaks]
  \frametitle{References}
  \bibliographystyle{abbrv}
  {\footnotesize \bibliography{bibliography} }
\end{frame}

%%--------------------------------%%


\end{document}



